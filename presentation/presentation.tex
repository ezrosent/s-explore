\documentclass{beamer}

\usetheme{bjeldbak}
\usepackage{minted}
% \usepackage[minted]{tcolorbox}
\renewcommand{\L}{\mathcal{L}}
\newcommand{\E}{\mathcal{E}}

\title{Semantics Exploration: Avoiding Surprise in Language Design}
\author{Eli Rosenthal}

\begin{document}
{%
  \setbeamertemplate{headline}{}
  \frame{\titlepage}
}

\tableofcontents

\section{Introduction}
\begin{frame}
  \frametitle{When are Languages Surprising?}
  
  \begin{itemize}
  \item A lot of programming languages have similar constructs. For various
    reasons, programmers build up certain models of how these constructs behave.
    \pause
  \item \emph{Personal Example:} I expect lexical scoping when I encounter a feature in a
    new language described as a ``lambda function.''
    \begin{itemize}
    \item Possibly because my first language was an HtDP teaching language,
      followed closely by ML.
    \item Also potentially because I gravitate to static scoping disciplines because
      they are more compositional than dynamic scope.
    \end{itemize}
    \pause
  \item Another common example is the extent to which standard arithmetic
    operators are not commutative or even associative in many popular
    programming languages, be it due to floating point, or ``OO''$+$ operator overloading.

\end{itemize}

\end{frame}

\begin{frame}
  \frametitle{Background/Related Work}
  {\small \it I wasn't sure there was any$\ldots$ but then Shriram told me to
    look at this paper}
  \begin{itemize}
  \item Felleisen's 1990 paper ``On the Expressive Power of Programming
    Languages.'' Provides a formal account of how expressive a language is based on
    operational semantics, as opposed to computability.

  \item Define a language $\L$ as a set of freely generated \emph{phrases} from
    a grammar, a subset of valid programs, a set of values, and an
    \emph{operational semantics}

  \item The above definition gives rise to a natural notion of
    \emph{sublanguage}, and \emph{syntactic abstraction}
    
   \item This provides a formal foundation for how expressive a language is,
     based on what features can be expressed in terms of some desugaring
     (skipping over many details here) that preserves behavioral equivalence.

   \item This set of definitions also (albiet implicitly) lays out conditions
     for features to be implemented orthogonally, 

  \end{itemize}
  
\end{frame}

\begin{frame}
  \frametitle{Roadmap}
  \begin{itemize}
  \item Surprise

    \begin{itemize}
    \item Formal/idealized Definition
    \item Practical implementation/tooling
    \end{itemize}

  \item Orthogonality (How might we avoid surprises?)

    \begin{itemize}
    \item Writing interpreters as \emph{compositions of features} that are
      orthogonal by costruction (and hence compositional).
    \end{itemize}
    
 
  \end{itemize}
  
\end{frame}

\section{Surprise: Theory and Practice}
\subsection{Theory}
\begin{frame}
  \frametitle{Surprise: the Ideal case}
  Consider two langauges, R, and the $\lambda$ calculus.
  \begin{itemize}
  \item In an idealized setting, assume these languages have some notion of
    syntax, and an operational semantics.
  \item Ask the question of ``to what extent is R's scoping surprising?''
    \begin{itemize}
    \item This implies a syntactic mapping $\E$ of the sort 
      \[
        \E(\texttt{function(a,b)d}) = \lambda a\,b.~\E(d)
      \]    
      Along with some natural inverse mapping. Note that this mapping can even
      be partial, implicitly introducing stuck states in $\lambda$; these act as
      uninterpreted values.

    \item If we can extend this syntactic mapping to a mapping of
      \emph{evaluation contexts}, then we can create a hybrid
      operational semantics that applies this mapping, takes some number of
      $\lambda$ evaluation steps, and maps it back to $R$.
      

    \end{itemize}
  \item If this new composite relation is nondeterministic, then there is some
    sequence of $\lambda$-calculus reduction steps that breaks with R's
    semantics.
  \end{itemize}
  
\end{frame}

\begin{frame}
  \frametitle{Limitations}
  \begin{itemize}

  \item I think this is actually a nice definition, and seems to fit into a
    natural extension into Felleisen's framework. It's really
    Felleisen+re-sugaring, allowing us to consider behavioral equivalence in the
    host language as opposed to the de-sugared language. {\small Assuming I'm
      understanding the paper right}

    \pause

  \item But it significantly limits what languages we can model; we would need
    an operational semantics for R, which is $\mathfrak{not~easy}$
    \pause
    
  \item[\emph{Idea}] Limit ourselves to syntactic mappings + evaluator for the big language,
    see how far we can get.
    
  \end{itemize}
\end{frame}
\subsection{Practice}

\begin{frame}[fragile]
  \frametitle{Algebraic Data-types and Recursion}
  We will operate on \emph{unrolled} algebraic data-types (ADTs), i.e. given a
  standard recursive ADT
\begin{minted}{haskell}
data Exp = Number Integer | Add Exp Exp
\end{minted}
We expect it to be given as
\begin{minted}{haskell}
data Exp a = Number Integer | Add a a
\end{minted}
Where we can recover a type isomorphic to the original \texttt{Exp} by tying the
recursive knot ourselves:
\begin{minted}{haskell}
newtype Mu f = Mu (f (Mu f))
type Exp' = Exp (Mu Exp)
\end{minted}

This grants us greater flexibility.
\end{frame}

\begin{frame}[fragile]
  \frametitle{Interleaving Two ADTs}

  This representation allows us to account for two ADTS where each term allows
  sub-terms of either type.
\begin{minted}{haskell}
newtype CR a b = CR (Either (a (CR a b)) (b (CR a b)))
type a :+: b = CR a b
\end{minted}

This lets us define easy syntactic mappings and fuzzing operations.

(Look at Code)

\end{frame}

\begin{frame}[fragile]
  \frametitle{Testing for violations}
  \begin{itemize}

  \item Given this mapping, we can subsitute terms in for referentially
    transparent ones.

  \item With the $\lambda$-calculus, this is some number of $\beta$/$\eta$ rule
    applications.
    
  \item This, along with \texttt{Arbitrary} instances, is fertile ground for
    automated counterexample finding!

  \end{itemize}
  
(Look at Code)

\end{frame}

\section{Orthogonality}

\begin{frame}[fragile]
  \frametitle{Avoiding Surprise}
  \begin{itemize}

  \item A related notion to surprise is that of \emph{orthogonality}, which (I
    think) is the idea that different language features do not interact in
    destructive ways.

  \item This implies some notion of composition: if a given features does
    not interfere with others, than evaluation should be the structural thing
    that we expect. 

  \item[\emph{Idea}] Reverse-engineer the syntactic approach from before to
    create features that are orthogonal by construction.

  \end{itemize}

\end{frame}
\begin{frame}[fragile]
  Consider this simple language:
\begin{minted}{haskell}
data Exp 
  = Id String 
  | Lam String Exp 
  | App Exp Exp 
  | ArithOp (Integer -> Integer -> Integer) Exp Exp 
  | N Integer
  | BoolOp (Bool -> Bool -> Bool) Exp Exp
  | B Bool
  | IF Exp Exp Exp
\end{minted}
\end{frame}

\begin{frame}[fragile]
  Then refactor it into its component features, and unwrapped:
  \begin{minted}{haskell}
data UTExp a  = Id String | Lam String a | App a
data ArithExp a  = N Integer | ArithOp (Integer -> Integer -> Integer) a
data BoolExp a  = B Boolean | BoolOp (Boolean -> Boolean -> Boolean) a
data Exp = AE ArithExp | BE BoolExp | UE UTExp

-- value type!
data Val = NV Integer | BV Boolean | Clos Exp
\end{minted}

\end{frame}

\begin{frame}[fragile]
  Then write an evaluator with the ``extensible let-rec trick''
\begin{minted}{haskell}
arithEval _    (N n)           = Just (NV n)
arithEval eval (ArithOp f a b) = do 
  a' <- eval a
  b' <- eval b
  case (a', b') of 
     (Just (NV x), Just (NV y)) -> return $ NV $ f x y
     _ -> Nothing
\end{minted}
\begin{minted}{haskell}
boolEval :: (a -> Maybe Val) -> BoolExp a -> Maybe Val
-- ...
\end{minted}
\begin{minted}{haskell}
utlcEval :: (a -> Maybe Val) -> UTExp a -> Maybe Val
-- ...
\end{minted}
\begin{minted}{haskell}
eval :: Exp -> Maybe val
eval (AE a) = arithEval eval a
eval (BE a) = boolEVal eval a
eval (UE a) = utlcEval eval a
\end{minted}

\end{frame}

\begin{frame}
  \frametitle{What's the Point?}
    \begin{itemize}
    \item Parametricity
      \begin{itemize}
      \item The evaluators for individual features cannot even inspect what
        features are in the languages.
      \item Orthogonality appears to be a free theorem!
      \end{itemize}
      \pause
    \item Though there are some scaling issues, a lot of this can be automated.
      \pause
    \item Also elucidates ``copmosition'' bent to earlier part of project. Note
      that 
      \[
        \texttt{Exp}  \simeq \texttt{ArithExp :+: UTExp :+: BoolExp}
      \]

      If some compositional language models the features of a larger one, then
      it effectively factors out into a ``bag-of-features'' data type, and
      quick-check is what ensures parametricity.
    \end{itemize}
  
\end{frame}
\begin{frame}
  
\centerline{That's about it! Questions?}
  
\end{frame}
\end{document}
